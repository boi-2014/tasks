\documentclass{boi2014-et}

\usepackage{enumitem}

\renewcommand{\DayNum}{2}
\renewcommand{\TaskCode}{portals}
\renewcommand{\TaskName}{Portaalid}

\newcommand{\constant}[1]{{\tt #1}}

\begin{document}

    \begin{wrapfigure}[4]{r}{4cm}
        \vspace{-24pt}
        \includegraphics[width=4cm]{\TaskCode.jpeg}
    \end{wrapfigure}

    Labürinti on peidetud kook, mida sa tahad ära süüa.
    Sul on labürindi kaart, mis on ruudustik $R$ rea ja $C$ veeruga.
    Ruudustiku iga element sisaldab üht järgmistest märkidest:
    \begin{description}[itemindent=1pt]
        \item[\constant{\#}] (ruut) tähistab seina,
        \item[\constant{.}] (punkt) tähistab tühja ruutu,
        \item[\constant{S}] (suur s) tähistab sinu asukohta,
        \item[\constant{C}] (suur c) tähistab kooki.
    \end{description}

    Sa võid kõndida ainult mööda tühje ruute ning liikuda ühelt ruudult teisele,
    kui neil on ühine külg. Lisaks on kaardil kujutatud ristkülikukujuline ala
    väljastpoolt ümbritsetud samasuguse seinaga.

    Et kiiremini koogini jõuda, on sul Aperture Science\texttrademark{}
    portaaliheitepüss, mis töötab järgmiselt:
    Igal hetkel saab sellega tulistada kas
    \emph{üles}, \emph{vasakule}, \emph{alla} või \emph{paremale}.
    Kui portaal mingis suunas tulistatakse, lendab see selles suunas, kuni tabab seina.
    Kui see juhtub, tekib seina sinupoolsele küljele portaal.

    Korraga saab eksisteerida kuni kaks portaali. Kui kaks portaali on juba olemas,
    saab ühe neist (sina otsustad, kumma) eemaldada, kasutades portaalipüssi uuesti.
    Tulistades portaalipüssist seina pihta, millel juba on portaal, asendatakse see portaal
    (iga seina samal küljel saab korraga olla ainult üks portaal).
    Seina erinevatel külgedel võivad portaalid korraga olla.

    Kui kaardile on kaks portaali paigutatud, saab neid kasutada enda teleporteerimiseks.
    Seistes ühe portaali kõrval, saab sellesse sisse kõndida ning väljuda ruudul,
    mis on teise portaali kõrval. See võtab sama palju aega kui kõndimine ühelt naaberruudult teisele.

    Võib eeldada, et portaalide tulistamine ei võta aega ning ühelt ruudult teisele liikumine
    (või läbi portaalide teleporteerimine) võtab ühe ühiku aega.

    \Task

    Labürindi kaardi, oma esialgse asukoha ning koogi asukoha põhjal leida
    minimaalne koogini jõudmiseks vajalik aeg.

    \Input

    Sisendi esimesel real on kaks täisarvu: kaardi ridade arv
    $R$ ning veergude arvu $C$. Järgmised $R$ rida kirjeldavad
    kaarti. Igal real on $C$ märki: \constant{\#},
    \constant{.}, \constant{S} või \constant{C} (kirjeldatud eespool).

    Märgid \constant{S} ja \constant{C} esinevad kaardil kumbki täpselt ühe korra.

    \Output

    Väljund peab koosnema ühest täisarvust --- minimaalne aeg,
    mis on vajalik algpositsioonist koogini jõudmiseks.

    Võib eeldada, et algpositsioonist saab alati koogini jõuda.

    \Example

    \example
    {
        4 4\newline
        .\#.C\newline
        .\#.\#\newline
        ....\newline
        S...
    }
    {
        4
    }
    {
        Üks võimalik lühim käikude järjekord on: 1) liigu paremale, 2) liigu paremale,
        tulista üks portaal üles ja teine alla, 3) liigu läbi
        alumise portaali (jõuad asukohta $rida = 0,
        veerg = 2$), 4) liigu paremale ja jõuad koogini.

        \begin{center}
            \includegraphics[width=4cm]{portals-example}
        \end{center}
    }

    \Scoring

    \begin{description}[leftmargin=0pt]
        \item[Alamülesanne 1 (? points):] $0 \le R \le 10, 0 \le C \le 10$.
        \item[Alamülesanne 2 (? points):] $0 \le R \le 50, 0 \le C \le 50$.
        \item[Alamülesanne 3 (? points):] $0 \le R \le 200, 0 \le C \le 200$.
            Iga ruudu kõrval on vähemalt üks sein.
        \item[Alamülesanne 4 (? points):] $0 \le R \le 200, 0 \le C \le 200$.
        \item[Alamülesanne 5 (? points):] $0 \le R \le 1000, 0 \le C \le 1000$.
    \end{description}

    \Constraints

    \begin{description}
        \item[Ajapiirang:] ? s.
        \item[Mälupiirang:] ? MB.
    \end{description}

\end{document}
