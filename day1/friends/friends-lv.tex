\documentclass{boi2014-lv}

\renewcommand{\DayNum}{1}
\renewcommand{\TaskCode}{friends}
\renewcommand{\TaskName}{Trīs draugi}
\renewcommand{\TaskVersion}{1.1}

\begin{document}
    \begin{wrapfigure}{r}{4cm}
        \vspace{-24pt}
		\includegraphics[width=4cm]{\TaskCode.jpeg}
	\end{wrapfigure}
		Trīs draugiem patīk spēlēt šādu spēli.
		Pirmais draugs izvēlas burtu virkni $S$. Pēc tam otrais draugs izveido jaunu burtu virkni $T$, kas sastāv no divām ori\v{g}inālās virknes $S$ kopijām. Beigās trešais draugs izveidotajā virknē iesprauž vienu burtu virknes $T$ sākumā, beigās vai pa vidu, tādā veidā izveidojot virkni $U$.
    %Three friends like to play the following game.
    %The first friend chooses a string.
    %Then the second friend constructs a new string that consists of
    %two copies of the original string. 
    %Finally, the third friend inserts one letter somewhere in the string.

    \Task
    Jums tiek iedota burtu virkne $U$ un jūsu uzdevums ir rekonstruēt sākotnējo virkni $S$.
		%You are given the final string and your task is to reconstruct the original
    %string.

    \Input
    Ievaddatu pirmā rinda satur beigās iegūtās burtu virknes garumu $N$. Beigās iegūtā burtu virkne dota ievaddatu otrajā rindā. Virkne sastāv no $N$ lielajiem burtiem A, B, C, \ldots{}, Z.
		%The first line of the input contains the length of the final string $N$.
    %The string itself is given on the second line. It consists of $N$
    %uppercase letters A, B, C, \ldots{} Z.

    \Output
		Jūsu programmai jāizvada ori\v{g}inālā burtu virkne, taču eksistē divi izņēmumi:
    %Your program should print the original string.
    %However, there are two exceptions:
    \begin{enumerate}
        \item Ja ievaddatos norādītā virkni nevarēja iegūt izmantojot iepriekš aprakstīto paņēmienu, tad Jūsu programmai jāizvada {\tt NOT POSSIBLE}.%If it is not possible that the final string is created using above
        %procedure, you should print {\tt NOT POSSIBLE}.
        \item Ja sākotnējā virkne nav viennozīmīgi nosakāma, tad Jūsu programmai jāizvada {\tt NOT UNIQUE}. %If the original string is not unique, you should print {\tt NOT UNIQUE}.
    \end{enumerate}
    

    \Examples

    \simpleexample{7\newline ABXCABC}{ABC}{}
    \simpleexample{6\newline ABCDEF}{NOT POSSIBLE}{}
    \simpleexample{9\newline ABABABABA}{NOT UNIQUE}{}

    \Scoring

    \begin{description}
        \item[Apakšuzdevums 1 (35 punkti):] $1 \le N \le 2001$.
        \item[Apakšuzdevums 2 (65 punkti):] $1 \le N \le 2000001$.
    \end{description}

    \Constraints

    \begin{description}
        \item[Laika ierobežojums:] 0.5 s.
        \item[Atmiņas ierobežojums:] 256 MB.
    \end{description}

\end{document}

