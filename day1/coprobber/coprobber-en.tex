\documentclass{boi2014}

\usepackage{enumitem}
\usepackage{todonotes}
\usepackage{wrapfig}

\renewcommand{\DayNum}{1}
\renewcommand{\TaskCode}{coprobber}
\renewcommand{\TaskName}{A cop and a robber}

\newcommand{\param}[1]{{\tt #1}}
\newcommand{\method}[1]{{\tt #1}}
\newcommand{\constant}[1]{{\tt #1}}

\begin{document}
    \todo{Wrap the problem in a story.}
    A cop is trying to catch a robber moving along the edges of a graph. The
    two players play alternatingly.

    \begin{enumerate}
        \item The game starts by the cop choosing his initial vertex.
        \item The robber then chooses his starting vertex.
        \item The cop's move consists of him moving to a neighbouring vertex
        (i.e.  a vertex that is connected by an edge to the current vertex of
        the cop) or waiting (i.e. not moving).
        \item The robber's move consists of him moving to a neighbouring
        vertex. Note that he is not allowed to wait.
        \item The robber's move consists of him moving to a neighbouring vertex.
        Note that he is not allowed to wait.
        \begin{enumerate}
            \item both players are on the same vertex after a move of either
            player --- cop captures the robber so wins;
            \item the position repeats (any position is dened by cop's vertex
            $c$, robber's vertex $r$ and the side $T$ whose turn it is to move
            next; a position repeats if there has been an earlier move after
            which all three parameters $c$, $r$ and $T$ were the same) --- this
            corresponds to the robber being able to avoid the cop indefinitely,
            so robber wins.
        \end{enumerate}
    \end{enumerate}

    \Task
    Given the graph, determine which player wins if both play optimally.
    Moreover, play the game for the winning player and win.

    \Implementation
    You need to implement the function \method{get\_cake(R, C, M)}
    which takes the following parameters:
    \begin{itemize}
        \item \param{R} --- the number of rows in the map $R$,
        \item \param{C} --- the number of columns in the map $C$,
        \item \param{M} --- a two--dimensional array, where
            \param{M}$[i][j]$ ($0 \le i < R$, $0 \le j < C$) is the cell in
             the $i$--th row and $j$--th column of the map.
    \end{itemize}


    \Example
    \todo{Provide an example.}

    \Scoring
    \todo{Explain scoring.}

    \todo{Prepare subtasks.}
    \begin{description}
        \item[Subtask 1 (25 points):]
        \item[Subtask 2 (25 points):]
        \item[Subtask 3 (25 points):]
        \item[Subtask 4 (25 points):]
    \end{description}

    \Constraints

    \begin{description}
        \item[Time limit:] 1 s.
        \item[Memory limit:] 256 MB.
    \end{description}

    \todo{Need to add info about graders.}
\end{document}

