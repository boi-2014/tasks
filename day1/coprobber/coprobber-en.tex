\documentclass{boi2014}

\usepackage{enumitem}
\usepackage{todonotes}
\usepackage{wrapfig}

\renewcommand{\DayNum}{1}
\renewcommand{\TaskCode}{coprobber}
\renewcommand{\TaskName}{The Cop and the Robber}

\newcommand{\param}[1]{{\tt #1}}
\newcommand{\method}[1]{{\tt #1}}
\newcommand{\constant}[1]{{\tt #1}}

\begin{document}
    In the city of Bytemore the crime level is hitting an all--time high.  Among
    other crimes, roberies are happening every day.  If a police officer happens
    to be nearby, he will try to chase the robber through the small alleys that
    connect street corners (usually referred to as just \emph{corners}).
    Unfortunately most of the time robbers escape as they know the district much
    better than the police.

    The Bytemore City Police Department (BCP) is organising a summit in order to
    reduce the crime rate. One of the initiatives is to use computer aid in
    pursuing the robbers. For this BCP has made precise maps of all dangerous
    city districts. Now they need computer software to find best chasing
    strategies.

    The pursuing problem is modelled as follows:
    \begin{enumerate}
        \item A police officer chooses a corner on which to patrol.
        \item A robber then chooses a corner for the robbery (robbers always
        know where police~is).
        \item The police officer's move consists of him moving to a neighbouring
        corner (i.e. one that is connected to the current one by an alley) or
        waiting (i.e. not moving).
        \item The robber's move consists of him moving to a neighbouring corner.
        Note that, unlike the police, robbers cannot wait. It is in
        their instinct to keep running.
        \item The police officer and the robber keep making moves one after
        another until one of the following happens:
        \begin{enumerate}
            \item a situation repeats itself (a situation is defined by police
            officer's position $P$, robber's position $R$ and the side $T$ whose
            turn it is to move next). This corresponds to the robber being able
            to avoid the police officer indefinitely, so the robber escapes;
            \item they both meet on the same corner after a move of either of
            them. In this case the police officer catches the robber.
        \end{enumerate}
    \end{enumerate}

    \Task
    You have to write a program which, given the plan of a city district, would
    find whether catching the robber is possible, and if that is the case, would
    choose the starting position for the police officer and make required moves
    in order to catch the robber.
    
    Your program must assume that both the police officer and the robber move
    optimally.

    \Implementation
    You need to implement following two functions:
    \begin{enumerate}
        \item \method{Start(N, M, U[], V[])} --- this function takes as
        parameters number of corners $N$ and number of alleys connecting those
        corners $M$. The arrays $U$ and $V$ are each of length $M$ and indicate
        that there is an alley connecting corners $U[i]$ and $V[i]$ for every $0
        \le i \le M - 1$.

        This function has to return a number $P$ ($1 \le P \le N$) --- the
        starting corner of the police officer. If it is impossible to catch the
        robber, the function has to return 0 (zero).

        \item \method{NextMove(R)} --- this function takes as a parameter the
        current corner of the robber $R$ and must return the corner of the
        police officer after his move.
    \end{enumerate}

    The function \method{Start} will be called exactly once before any
    calls to \method{NextMove}. If it returns a non--zero, the function
    \method{NextMove} will be called zero or more times.

    The program will terminate as soon as one of the following happens:
    \begin{itemize}
        \item \method{NextMove} returns an invalid move.
        \item A situation repeats itself.
        \item The robber is caught.
    \end{itemize}

    The corners are numbered from $1$ to $N$. You can assume that the robber
    always makes valid moves, i.e. moves from one corner to another if there's
    an alley connecting the two, and never waits.

    \Example
    \todo{Provide an example.}

    \Scoring
    \todo{Explain scoring.}

    \todo{Prepare subtasks.}
    \begin{description}
        \item[Subtask 1 (25 points):]
        \item[Subtask 2 (25 points):]
        \item[Subtask 3 (25 points):]
        \item[Subtask 4 (25 points):]
    \end{description}

    \Constraints

    \todo{Check constraints.}
    \begin{description}
        \item[Time limit:] 1 s.
        \item[Memory limit:] 256 MB.
    \end{description}

    \todo{Need to add info about graders.}
\end{document}

