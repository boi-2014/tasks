\documentclass{boi2014-lv}

\usepackage{todonotes}

\renewcommand{\DayNum}{1}
\renewcommand{\TaskCode}{sequence}
\renewcommand{\TaskName}{Virkne}

\begin{document}
    \begin{wrapfigure}[5]{r}{5cm}
        \vspace{-24pt}
		\includegraphics[width=5cm]{\TaskCode.jpeg}
	\end{wrapfigure}

		Ādams uz tāfeles uzrakstīja virkni no $K$ secīgiem pozitīviem skaitļiem sākot no $N$. Kad viņš aizgāja, Billijs no katra uzrakstītā skaitļa nodzēsa visus izņemot vienu ciparu, tādējādi uz tāfeles palika virkne no $K$ viencipara skaitļiem. 
    %Adam wrote down a sequence of $K$ consecutive positive integers starting
    %with $N$ on a blackboard. When he left, Billy came in and erased all but one
    %digit from each number, thus creating a sequence of $K$ integers between 0
    %and 9.

    \Task

		Dotai iegūtajai virknei atrodiet mazāko korekto $N$ vērtību, kura varēja tikt izmantota kā virknes pirmais skaitlis.
    %Given the final sequence left on the blackboard, find the smallest
    %value of $N$ with which it could have occured.

    \Input

		Ievaddatu pirmā rinda satur vienu veselu pozitīvu skaitli $K$ --- virknes garumu. Otrā rinda satur $K$ skaitļus $B_1$, $B_2$, \dots, $B_K$ --- Billija iegūto virkni ($0 \le B_i \le 9$),  tādā secībā kādā tā bija rakstīta uz tāfeles.
		
    %The first line of the input contains a single integer $K$ --- the length of
    %the sequence. The second line contains $K$ integers $B_1$, $B_2$, \dots,
    %$B_K$ --- Billy's sequence, in the order in which it is written on the
    %blackboard.

    \Output

		Izvaddatu vienīgajā rindā jāizvada mazākā $N$ vērtība ar kuru sākotnējā virkne varētu būt sākusies.
    %The output should consist of a single line with the smallest value of
    %$N$ with which this sequence could have occured.

    \Example

    \example
    {
        6\newline
        7 8 9 5 1 2
    }
    {
        47
    }
    {
				$N = 47$, kuram atbilstu Ādama virkne $47\ 48\ 49\ 50\ 51\ 52$ no kuras ir iespējams izveidot Billija iegūto virkni. Tā kā neviena cita mazāka $N$ vērtība nebūtu derējusi, atbilde ir 47.
        %$N = 47$ would correspond to Adam's sequence
        %being $47\ 48\ 49\ 50\ 51\ 52$ from which Billy's sequence
        %can indeed be obtained. As no smaller value of $N$
        %would work, the answer is 47.
    }

\Scoring

\begin{description}
    \item[Apakšuzdevums 1 (9 punkti).] $1 \le K \le 1000$, pareizā atbilde nepārsniedz $1000$ %correct
        %answer does not exceed $1000$
    \item[Apakšuzdevums 2 (33 punkti).] $1 \le K \le 1000$
    \item[Apakšuzdevums 3 (25 punkti).] $1 \le K \le 100\,000$, visi dotās virknes locekļi ir vienādi%all
		%elements of the given sequence are equal
    \item[Apakšuzdevums 4 (33 punkti).] $1 \le K \le 100\,000$
\end{description}

\Constraints

Laika ierobežojums: $1$ s.

Atmiņas ierobežojums: $256$ MB.

\end{document}

