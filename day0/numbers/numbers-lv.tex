\documentclass{boi2014-lv}

\usepackage{enumitem}
\usepackage{todonotes}

\renewcommand{\DayNum}{0}
\renewcommand{\TaskCode}{numbers}
\renewcommand{\TaskName}{Skaitļi}

\begin{document}

    Doti divi naturāli skaitļi, atrodiet visus tos naturālos skaitļus, kas atrodas intervālā starp tiem neieskaitot galapunktus.
		%Given two positive integers, find all integers that are strictly
    %between them.

    \Input
    Ievaddatu vienīgajā rindā doti divi naturāli skaitļi $A$ un $B$.
		%Input consists of a single line that contains two positive
    %integers $A$ and $B$.

    \Output
		Pirmajā izvaddatu rindā jābūt vienam veselam skaitlim --- naturālo skaitļu skaits intervālā starp $A$ un $B$ neieskaitot galapunktus. Ja intervālā starp $A$ un $B$ ir vismaz viens naturāls skaitlis, tad otrajā rindā jāizvada visi šie skaitļi augošā secībā.
    %The first line of output should consist of a single integer ---
    %the number of integers that are strictly between $A$ and $B$.
    %If there is at least one such integer, then all of them should
    %be listed in increasing order in the second line of output.
    
    \Examples

    \simpleexample
    {
        8 14
    }
    {
        5 \newline
        9 10 11 12 13
    }

    \simpleexample
    {
        1 2
    }
    {
        0
    }

    \Scoring

				Jūsu iesūtījums iegūs visus punktus par apapkšuzdevumu, ja tas izvadīs pareizu rezultātu visiem apakšuzdevuma testiem. Ja pareizi visiem apapkšuzdevuma testiem būs tikai pirmajā rindā izvadītie dati, tad risinājums iegūs $50\%$ no apakšuzdevuma punktiem. Citos gadījumos risinājums neiegūs punktus par apakšuzdevumu.
				Ievērojiet, ka punkti netiks piešķirti par apakšuzdevumu, ja kaut viens no apakšuzdevuma testiem pārsniedz laika ierobežojumus vai notiek izpildes laika kļūda.
        %Your submission will score full points for a subtask if it
        %produces correct output for each testcase in 
        %that subtask. Otherwise, if it produces correct first
        %line of output for each testcase, it will score $50\%$ of the
        %points for the subtask. In all other events it will score no
        %points for the subtask.
        %Note that no points will be awarded for a subtask if
        %(in particular) time
        %limit is exceeded or a runtime error is encountered in at
        %least one testcase in that subtask.

    \begin{description}

        \item[Apakšuzdevums 1 (30 punkti):] $1 \le A, B \le 1000$. 
        \item[Apakšuzdevums 2 (70 punkti):] $1 \le A, B \le 10^{15}$,
            $A$ un $B$ atšķiras ne vairāk kā par $100\,000$.
    \end{description}

    \Constraints

    \begin{description}
        \item[Laika ierobežojums:] 1 s.
        \item[Atmiņas ierobežojums:] 64 MB.
    \end{description}

\end{document}

