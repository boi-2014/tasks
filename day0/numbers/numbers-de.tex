\documentclass{boi2014-de}

\usepackage{enumitem}
\usepackage{todonotes}

\renewcommand{\DayNum}{0}
\renewcommand{\TaskCode}{numbers}
\renewcommand{\TaskName}{Zahlen}

\begin{document}

    Gegeben seien zwei positive Integer-Zahlen.  
    Finde alle Integer, die echt zwischen den gegebenen Zahlen liegen.

    \Input
    Die einzige Zeile der Eingabe enthält zwei positive Integer $A$ und $B$.

    \Output
    Die erste Ausgabezeile sollte genau ein Integer enthalten: 
    die Anzahl der Integer, die echt zwischen $A$ und $B$ liegen.
    Wenn es mindestens ein solches Integer gibt, sollten alle solche
    in der zweiten Zeile ausgegeben werden, und zwar in aufsteigender Folge.
    
    \Examples

    \simpleexample
    {
        8 14
    }
    {
        5 \newline
        9 10 11 12 13
    }

    \simpleexample
    {
        1 2
    }
    {
        0
    }

    \Scoring

        In jeder Subtask werden die Punkte so vergeben:
        Die volle Punktzahl wird erreicht,
        wenn die Ausgabe für alle Testfälle korrekt ist.
        Andernfalls, wenn immer die erste Ausgabezeile korrekt ist, 
        werden $50\%$ der Punkte erreicht.  Sonst werden keine Punkte erreicht.
        Beachte, dass es auch keine Punkte gibt, wenn in mindestens einem
        Testfall der Subtask das Time oder Memory Limit überschritten wird 
        oder ein Laufzeitfehler passiert.

    \begin{description}

        \item[Subtask 1 (30 Punkte):] $1 \le A, B \le 1000$. 
        \item[Subtask 2 (70 Punkte):] $1 \le A, B \le 10^{15}$,
            $A$ und $B$ unterscheiden sich um höchstens $100\,000$.
    \end{description}

    \Constraints

    \begin{description}
        \item[Time Limit:] 1 s.
        \item[Memory Limit:] 64 MB.
    \end{description}

\end{document}

